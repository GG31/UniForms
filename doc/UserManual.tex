\documentclass[a4paper,11pt,final]{report}
\usepackage[utf8]{inputenc} % Prendre en compte les caractères accentués
\usepackage[francais]{babel} % Prendre en compte les particularités de la typographie française.
\usepackage{geometry}         % marges
\usepackage{graphicx}         % images
\usepackage{setspace}
\usepackage[french]{varioref}
\usepackage{titlesec}  %titlespacing

\titlespacing{\chapter}{0pt}{*-5}{*5}
\titlespacing{\section}{0pt}{*2}{*2}
\titleformat{\chapter}[hang]{\bf\huge}{\thechapter}{2pc}{}
%\titleformat{\chapter}[hang]{\bf\huge}{\thechapter}{14pt}{\LARGE}
\renewcommand{\baselinestretch}{1.2}
\setlength{\parskip}{1.5ex plus .4ex minus .4ex}
\setlength{\parindent}{15pt} 
%\setlength{\topmargin}{-35pt}
%\setlength{\textheight}{600pt}


\title{\textbf{Manuel d'utilisation}\\E-Formulaire}
\author{}
\date{}
\begin{document}

\maketitle
\setcounter{page}{2}
\tableofcontents 
%Se mettre au niveau d'un non informaticien
%Expliquer les enchainements de fenêtre, erreurs
%FAQ
\chapter{Généralités}
\section{But de l'application}
Uniforms a pour but de fournir une plateforme de gestion de formulaires. C'est-à-dire, proposer des fonctionnalités de création et de soumission de formulaires ainsi que la possibilité de répondre à ces derniers et de consulter les réponses par le créateur du formulaire.

\section{Public visé}
Ce manuel est destiné aux personnes désirant créer un formulaire, le soumettre à une ou plusieurs personnes et consulter les réponses faites par les destinataires, mais aussi aux personnes répondant aux formulaires envoyés.

\section{Installation}

\section{Support technique}

\chapter{Fenêtre principale}
\section{Authentification}
\section{Page d'accueil}

\chapter{Création}
\section{Créer un formulaire}
\section{Modifier un formulaire}
\section{Consulter les réponses d'un formulaire}

\chapter{Répondre}
\section{Les formulaires dont je suis destinataires}
\section{Répondre à un formulaire}
\section{Modifier ma réponse}


\chapter{FAQ}
\section{Où sont les formulaires auxquels je peux répondre ?}
\section{Puis-je modifier la réponse d'un formulaire validée ?}

\end{document}
