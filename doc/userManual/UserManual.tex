\documentclass[a4paper,11pt,final]{report}
\usepackage[utf8]{inputenc} % Prendre en compte les caractères accentués
\usepackage[francais]{babel} % Prendre en compte les particularités de la typographie française.
\usepackage{geometry}         % marges
\usepackage{graphicx}         % images
\usepackage{setspace}
\usepackage[french]{varioref}
\usepackage{titlesec}  %titlespacing
\usepackage{caption}

\titlespacing{\chapter}{0pt}{*-5}{*5}
\titlespacing{\section}{0pt}{*2}{*2}
\titleformat{\chapter}[hang]{\bf\huge}{\thechapter}{2pc}{}
%\titleformat{\chapter}[hang]{\bf\huge}{\thechapter}{14pt}{\LARGE}
\renewcommand{\baselinestretch}{1.2}
\setlength{\parskip}{1.5ex plus .4ex minus .4ex}
\setlength{\parindent}{15pt} 
%\setlength{\topmargin}{-35pt}
%\setlength{\textheight}{600pt}


\title{\textbf{Manuel d'utilisation}\\E-Formulaire}
\author{}
\date{}
\begin{document}

\maketitle
\setcounter{page}{2}
\tableofcontents 
%Se mettre au niveau d'un non informaticien
%Expliquer les enchainements de fenêtre, erreurs
%FAQ
\chapter{Généralités}
\section{But de l'application}
Uniforms a pour but de fournir une plateforme de gestion de formulaires. C'est-à-dire, proposer des fonctionnalités de création et de soumission de formulaires ainsi que la possibilité de répondre à ces derniers et de consulter les réponses par le créateur du formulaire.

\section{Public visé}
Ce manuel est destiné aux personnes désirant créer un formulaire, le soumettre à une ou plusieurs personnes et consulter les réponses faites par les destinataires, mais aussi aux personnes répondant aux formulaires envoyés.

\section{Installation}

\section{Support technique}

\chapter{Fenêtre principale}
\section{Authentification}
La première page affichée est celle de l'authentification, que l'on peut voir figure~\ref{authentificationChoix}, elle en propose deux types : CAS (pour les utilisateurs disposants d'un identificant CAS commençant par les initiales du nom et prénom puis du numéro étudiant) ou autres utilisateurs.\\
%\includegraphics[width=.7\paperwidth]{images/authentification.png}\label{fig:ArgSimple}
\noindent\begin{minipage}{\linewidth}% to keep image and caption on one page
\makebox[\linewidth]{%        to center the image
  \includegraphics [width=150mm]{images/authentification.png}}
\captionof{figure}{Page de choix d'authentification}\label{authentificationChoix}
\end{minipage}

Pour accéder à la page d'authentification, il suffit de cliquer sur ``Utilisateurs CAS'' ou ``Autres utilisateurs''
\subsection{Utilisateur CAS}
Une fois choisi l'authentification CAS, la page figure~\ref{authentificationCAS} s'affiche.\\
Dans le champs ``Identification'', mettre votre identifiant (ex : dj209832, pour Jean Dupont ayant pour numéro étudiant 209832).\\
Entrer le mot de passe puis cliquer sur ``SE CONNECTER''. Si l'authentification échoue, une alerte ``Mauvais identifiant / mot de passe'' apparaitra pour vous en informer. Si elle réussit, la page d'accueil, figure~\ref{pageAccueil}, s'affichera. 

\noindent\begin{minipage}{\linewidth}% to keep image and caption on one page
\makebox[\linewidth]{%        to center the image
  \includegraphics [width=150mm]{images/authCAS.png}}
\captionof{figure}{Authentification CAS}\label{authentificationCAS}
\end{minipage}
\subsection{Autres utilisateurs}

\section{Page d'accueil}
Après authentification, la page figure~\ref{pageAccueil} s'affiche. On peut noter trois zones principales, une barre d'option, une zone de formulaires créés, une zone de formulaires à répondre.\\
\noindent\begin{minipage}{\linewidth}% to keep image and caption on one page
\makebox[\linewidth]{%        to center the image
  \includegraphics [width=190mm]{images/pageAccueil.png}}
\captionof{figure}{Page d'accueil}\label{pageAccueil}
\end{minipage}

\subsection{La barre d'options}
La barre d'options figure~\ref{barreOptions} propose divers boutons : ``Uniforms'', ``Créer un formulaire'', ``Mes Archives'' et ``Ayoub''. Les actions au clique de ces boutons sont les suivantes :
\begin{description}
	\item Uniform : Retour à la page d'accueil
	\item Créer un formulaire : Redirection vers la page de création d'un nouveau formulaire
	\item Mes archives : Redirection vers les formulaires archivés
	\item Ayoub : Identifiant de la personne dont la session est ouverte, au clique l'option ``Logout'' s'affiche pour la déconnection.
\end{description}

\noindent\begin{minipage}{\linewidth}% to keep image and caption on one page
\makebox[\linewidth]{%        to center the image
  \includegraphics [width=140mm]{images/barreOptionsAccueil.png}}
\captionof{figure}{Barre d'options}\label{barreOptions}
\end{minipage}

\subsection{Zone de formulaires créés}
La zone de formulaires créés liste l'ensemble des formulaires créés par l'utilisateur courant, les validés et non validés, figure~\ref{zoneFormCree}.\\
Un formulaire validé ou non peut être supprimé en cliquant sur la corbeille de la colonne ``Supprimer'' correspondante.

\noindent\begin{minipage}{\linewidth}% to keep image and caption on one page
\makebox[\linewidth]{%        to center the image
  \includegraphics [width=140mm]{images/zoneFormulairesCreesAccueil.png}}
\captionof{figure}{Liste des formulaires créés}\label{zoneFormCree}
\end{minipage}

\subsubsection{Les formulaires validés}
Les formulaires validés ne peuvent plus être modifié, ils ont été envoyés aux destinataires. Les seules actions possibles sur ces formulaires sont ``Voir résultats'' qui montre les réponses du formulaire qui ont déjà été faites par les destinataires et ``Supprimer''. On peut également obtenir l'url en cliquant sur le bouton ``URL'' de ce formulaire afin de l'envoyer par mail à un destinataire.

\noindent\begin{minipage}{\linewidth}% to keep image and caption on one page
\makebox[\linewidth]{%        to center the image
  \includegraphics [width=40mm]{images/urlForm.png}}
\captionof{figure}{URL d'un formulaire validé}\label{urlForm}
\end{minipage}

\subsubsection{Les formulaires non validés}
Les formulaires non validés n'ont pas encore été envoyés aux destinataires, ils peuvent donc être modifiés en cliquant sur le lien ``Modifier'' de la ligne du formulaire souhaité. Cette action mènera à la page ...???

\subsection{Zone de formulaires reçus}
Dans cette zone figure~\ref{zoneFormRecu} est listé l'ensemble des formulaires dont l'utilisateur courant est destinataire, c'est-à-dire ceux pour lesquels il doit répondre.\\
La liste des réponses en cours sont en bleus. Certains formulaires autorisent plusieurs réponses, c'est pourquoi on verra les lignes ``Réponse\string:0'', ``Réponse\string:1'', ``Réponse\string:2'' figure~\ref{zoneFormRecu} lorsque celles-ci sont initiés mais non validées.\\
Lorsque toutes les réponses autorisées ne sont pas initiées et/ou envoyées, le nombre restant est indiqué en vert, par exemple le formulaire 6 figure~\ref{zoneFormRecu} autorise 2 réponses en plus de celle initiée ``Réponse\string:0'' de la ligne juste en dessous.\\
Une fois toutes les réponses envoyés, seule la ligne du formulaire est présente, par exemple le formulaire 5 de la figure figure~\ref{zoneFormRecu}.\\

Les liens ``Modifier'' permettent la modification d'une réponse déjà commencée. Le lien ``Nouvelle réponse'' permet de commencer une réponse à un formulaire.

\noindent\begin{minipage}{\linewidth}% to keep image and caption on one page
\makebox[\linewidth]{%        to center the image
  \includegraphics [width=150mm]{images/zoneDestinataire.png}}
\captionof{figure}{Zone de formualaires reçus}\label{zoneFormRecu}
\end{minipage}

\chapter{Création}
\section{Créer un formulaire}
\section{Modifier un formulaire}
\section{Consulter les réponses d'un formulaire}

\chapter{Répondre}
\section{Les formulaires dont je suis destinataires}
\section{Répondre à un formulaire}
\section{Modifier ma réponse}


\chapter{FAQ}
\section{Où sont les formulaires auxquels je peux répondre ?}
\section{Puis-je modifier la réponse d'un formulaire validée ?}

\end{document}
